\chapter{Inleiding}\label{ch:inleiding}
Dit afstudeerverslag is het resultaat van de afstudeeropdracht welke is uitgevoerd in opdracht voor het bedrijf Eaglescience. Sinds 2009 richt Eaglescience zich op de ontwikkeling van complexe software oplossingen voor haar klanten op projectbasis.

Eaglescience hecht belang aan de ontwikkeling van zo veilig mogelijke software oplossingen voor haar klanten. Tijdens de ontwikkeling van applicaties maakt Eaglescience gebruik van externe bibliotheken, waardoor kwetsbaarheden in applicaties zouden kunnen worden geïntroduceerd. Eén van de manieren om software veiliger te maken is door het uitvoeren van ‘Software of Unknown Provenance’ analyses, ook wel SOUP-analyses genoemd. Deze analyses maken kwetsbaarheden in externe software inzichtelijk, zodat hierop actie kan worden ondernomen. Groei binnen Eaglescience en de hierdoor steeds groter wordende variabiliteit aan applicaties heeft ertoe geleidt dat er behoefte is ontstaan om SOUP-analyses op een geautomatiseerde manier op periodieke basis uit te voeren. De centrale onderzoeksvraag van deze afstudeeropdracht luidt; ‘Hoe kan Eaglescience middels een geautomatiseerde methode inzicht krijgen in potentiele kwetsbaarheden van gebruikte bibliotheken binnen projecten, waarbij rekening gehouden wordt met de huidige manier van werken?’.

De afstudeeropdracht is een aantal fasen uitgevoerd welke ieders in een eigen deel worden beschreven:
Deel~\ref{prt:opdracht} beschrijft het onderzoek naar Eaglescience zelf, hun werkwijze en de opdracht voor het afstudeerproject. In deel~\ref{prt:Onderzoek} wordt het onderzoeksplan besproken waarin middels een probleemanalyse, stakeholdersanalyse, en een conceptueel model naar een onderzoeksvraag met daaruit volgende deelvragen wordt gewerkt die vervolgens worden uitgewerkt en onderzocht. Het onderzoek zelf is ingedeeld in een theoretisch hoofdstuk waarin wordt onderzocht welke rol Software of Unkown provenance heeft in het ontwikkelen van software, welke gevaren het gebruik hiervan met zich mee kan brengen alsook welke instellingen zich bezig houden met het vastleggen van deze gevaren. Vervolgens wordt er gezocht naar tooling en een methode die binnen Eaglescience gebruikt kan worden om te achterhalen of er zich daadwerkelijk gevaren bevinden in de door Eaglescience ontwikkelde software. De uitkomst van dit laatste onderzoek is de basis voor het ontwerp van een applicatie welke in deel~\ref{prt:ontwerp} is beschreven. In dit deel wordt het ontwerp van de applicatie beschreven beginnend met de architectuur om vervolgens een aantal belangrijke onderdelen functioneel toe te lichten. Het afstudeerverslag wordt afgesloten door de deel~\ref{prt:conclusie}, waarin de eindconclusie wordt gegeven.



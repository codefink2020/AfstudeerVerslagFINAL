\pdfbookmark[1]{Inleiding}{Inleiding} % Bookmark name visible in a PDF viewer

\chapter{Inleiding}\label{ch:inleiding}

Voor u ligt het afstudeerverslag waarin de opdracht en de uitwerking hiervan wordt beschreven over het analyseren van $'$Software of Unkown Provenance$'$, in de praktijk ook wel SOUP-analyses genoemd. De opdracht is uitgevoerd voor het bedrijf Eaglescience wat zich sinds 2009 bezig houd met het ontwikkelen van complexe software oplossingen op Projectbasis.

Voor Eaglescience is het van belang dat er voor de klanten een zo veilig mogelijke oplossing wordt ontwikkeld. Eén van de manieren om software veiliger te maken is het inzichtelijk krijgen van de gebruikte bibliotheken bij het ontwikkelen van een aaplicatie. Doordat Eaglescience op project basis werkt er dus een grote variatiet aan oplossingen worden geboden is er de behoeft om dit inzicht op een geautomatiseerde manier te verkrijgen. En omdat software en externe bibliotheken zijn hier geen uitzondering, altijd worden vernieuwd of aangepast moet dit inzicht periodiek worden bijgesteld.


De opdracht is een aantal fasen uitgevoerd welke iedere in een eigen deel worden beschreven:
Deel~\ref{prt:opdracht} beschrijft het onderzoek naar Eaglescience zelf en welke werkwijze het hanteert. Daarna wordt er een onderzoeksplan opgesteld waarin middels een probleem analyse, stakeholdersanalyse, en een conceptueel model naar een hoofdvraag met daaruit volgend deelvragen gewerkt die vervolgens in Deel~\ref{prt:Onderzoek} worden onderzocht Het onderzoek zelf is ingedeeld in een theoretisch hoofdstuk waarin wordt onderzocht welke rol Software of Unkown provenance heeftin het ontwikkelen van software, welke gevaren het gebruik met zich mee kan brengen als ook welke instellingen zich bezig houden met het vastleggen van deze gevaren. Vervolgens wordt er gezocht naar een methode die binnen Eaglescience gebruikt kan worden om te achterhalen of er daadwerkelijk gevaren bevinden in de door Eaglescience ontwikkelde software. De uitkomst van dit laatste onderzoek is de basis voor het ontwerp van een applicatie  welke in deel~\ref{prt:ontwerp} is beschreven. In dit deel wordt het ontwerp van de applicatie beschreven beginnend met de architectuur om vervolgens een aantal belangrijke onderdelen functioneel toe te lichten.
Het deel~\ref{prt:conclusie} zal de conclusie van het onderzoek en ontwerp concluderen.




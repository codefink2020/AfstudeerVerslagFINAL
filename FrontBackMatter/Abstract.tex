% Abstract

%\renewcommand{\abstractname}{Abstract} % Uncomment to change the name of the abstract

\pdfbookmark[1]{Samenvatting}{Samenvatting} % Bookmark name visible in a PDF viewer

\chapter{Samenvatting}\label{ch:samenvatting}

De in dit afstudeerverslag beschreven afstudeeropdracht is uitgevoerd in opdracht van Eaglescience, een bedrijf welke software op maat ontwikkeld voor een grote verscheidenheid aan klanten. Tijdens de ontwikkeling van software maakt Eaglescience gebruik van externe bibliotheken, waardoor er potentiële gevaren zouden kunnen worden geïntroduceerd in de software. Om deze gevaren inzichtelijk te maken, zodat hierop actie kan worden ondernomen, wil Eaglescience een geautomatiseerde analyse methode inzetten. De centrale onderzoeksvraag luidt; ‘Hoe kan Eaglescience middels een geautomatiseerde methode inzicht krijgen in potentiële kwetsbaarheden van gebruikte bibliotheken binnen projecten, waarbij rekening gehouden wordt met de huidige manier van werken?’. In dit afstudeerverslag worden de onderzoeken beschreven die zijn uitgevoerd met als doel om een applicatie te ontwerpen die geautomatiseerd en periodiek de kwetsbaarheden van externe bibliotheken, oftewel $"$software of unknown provencance$"$ (SOUP) in kaart kan brengen.

Tijdens het onderzoek zijn de eisen die de opdrachtgever aan deze applicatie stelt in kaart gebracht, en is er onderzocht welke tooling en methodiek het beste aansluit bij de huidige werkwijze, Dev-stack en tooling van Eaglescience. Het onderzoek wees uit dat Eaglescience werkt met Scala en TypeScript, waarbij SBT en NPM als buildtools worden gebruikt. Door ontwikkeling in Scala bleek de beschikbare tooling voor SOUP analyses gelimiteerd voornamelijk doordat het een niche taal is die niet veelvuldig wordt gebruikt. Op basis van deze resultaten kan worden geconcludeerd dat de OWASP-dependency-check (voor NPM) en de sbt-dependency-check (voor SBT) het beste gebruikt kunnen worden voor de specifieke situatie van Eaglescience.
Na uitvoering van een reeks testen met deze tooling gericht op toepasbaarheid binnen de dev-stack van Eaglescience en de bruikbaarheid van de resultaten bleek dat deze geschikt zijn voor implementatie in de te ontwerpen applicatie.

Uit deze onderzoeken kan worden geconcludeerd dat de geselecteerde tooling inderdaad compatibel is met de specifieke werksituatie van Eaglescience. Op basis van deze conclusies is een ontwerp gemaakt voor een applicatie die geautomatiseerd en op periodieke basis SOUP analyses uit zou kunnen voeren. De ontworpen applicatie voldoet aan de eisen gesteld door de opdrachtgever en zou binnen de huidige Dev-stack kunnen worden geïmplementeerd. Tevens is er ruimte in het ontwerp voor toekomstige uitbreidingen. Wanneer Eaglescience de in dit afstudeerverslag ontworpen applicatie zou ontwikkelen zou deze hun in staat stellen om automatisch en periodiek SOUP analyses uit te kunnen voeren. Door het hierdoor verkregen inzicht in de aanwezigheid van mogelijke kwetsbaarheden in software te benutten zou Eaglescience op termijn veiligere producten kunnen leveren aan haar klanten.


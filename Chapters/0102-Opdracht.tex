% Chapter 2

\chapter{Opdracht}\label{ch:opdracht} % Chapter title
Tegenwoordig zijn software-bibliotheken niet meer weg te denken uit het huidige software-ontwikkelproces. Bibliotheken geven ontwikkelaars de mogelijkheid code te hergebruiken in meerdere projecten, om zo efficiënter te kunnen ontwikkelen. Dit helpt op zijn beurt om de time-to-market te verkorten. Bibliotheken kunnen door bedrijven zelf geschreven worden, in het geval van Eaglescience is dit ArchES, of worden overgenomen van andere bedrijven/instellingen. ArchES is echter zelf ook afhankelijk van een aantal bibliotheken die niet ontwikkeld zijn door Eaglescience. Hierdoor kan niet worden voorkomen dat bibliotheken worden gebruikt waarvan de afkomst niet geheel kan worden herleid.

Deze (deels) onherleidbare bibliotheken vallen onder de noemer $"$Software of Unknown Provenance / Pedigree (SOUP)$"$ ~\citep{Bischop:2001}. Door het gebruik van dit soort bibliotheken kan er een aannemelijk risico ontstaan op het gebied van kwetsbaarheden. Om inzicht te verkrijgen in deze kwetsbaarheden en daarmee mogelijke veiligheidsissues dient een SOUP-analyse gedaan te worden. Binnen Eaglescience wordt het belang hiervan onderstreept en daarom wordt er gezocht naar een efficiënte en, waar mogelijk, geautomatiseerde manier voor het uitvoeren van een dergelijke analyse. Hierdoor kan de veiligheid van de ontwikkelde applicaties worden gewaarborgd zonder afbreuk te doen aan kwaliteit.

\section{Opdracht vanuit Eaglescience}\label{sec:opdracht-vanuit-Eaglescience}

Vanuit de CTO is de wens ontstaan om een systematisch opgebouwde methode te ontwikkelen waarbij er automatisch en periodiek een SOUP-analyse kan worden gedaan op bestaande en nieuwe projecten. Het beoogde resultaat is een module die wordt toegevoegd aan de al bestaande portal van Eaglescience waarbij project verantwoordelijken inzicht kunnen verkrijgen in de kwetsbaarheden die in een project aanwezig kunnen zijn door het gebruik van externe bibliotheken.

De aanleiding van deze opdracht is een gebrek aan inzicht in reeds draaiende software. In tegenstelling tot software waaraan nog ontwikkeld wordt, gaat er bij projecten waaraan niet actief wordt ontwikkeld het inzicht in kwetsbaarhden verloren. Hierdoor is het onbekend welke kwetsbaarheden er mogelijk aanwezig zijn in deze projecten.
\subsection{Eisen aan de opdracht}\label{subsec: eisen-aan-de-opdracht}
Vanuit Eaglescience worden er een aantal eisen gesteld waaraan het eindproduct moet voldoen. Als er aan deze eisen is voldaan is er voor Eaglescience een waardevol product ontwikkeld welke in gebruik kan worden genomen. Ook is er aangegeven dat er bij voorkeur naar open source tooling  gekeken dient te worden, daar er geen budget voor aanschaf van software voor handen is.
\newpage

\textbf{Functionele eisen}
\begin{itemize}
\item De module dient eenvoudig te kunnen worden gebruikt in de huidige Continuous Integration /Continuous Deployment (CI/CD) pipeline voor bestaande en nieuwe projecten
\item De module dient gebruik te maken van de bestaande huidige projectstructuur van de portal
\item De module dient ondersteuning te bieden aan meerdere omgevingen (OTAP)
\item De module dient te worden ontwikkeld in Angular en Play (Scala), zodat het in de bestaande portal module past
\item De module dient met een instelbaar interval de analyse uit te voeren
\item De module moet op project en omgeving niveau rapporteren over bekende kwetsbaarheden
\item De module dient kwetsbaarheden op minimaal drie niveau’s in te schalen (kritisch, gemiddeld en laag)
%\item De module dient ondersteuning te bieden voor het instellen van quality gates ten aanzien van de melding die het vind van ieder niveau, per project, per omgeving
\end{itemize}
\textbf{Kwaliteitseisen}
\begin{itemize}
\item De module dient te voldoen aan de geldende kwaliteitsnormen binnen Eaglescience, minimaal meetbaar door:
	\begin{itemize}
	\item Test coverage > 70\%
	\item Onderdeel van de bestaande CI/CD voor de Eaglescience Portal
	\end{itemize}
\item De geschreven code dient gereviewd te worden door een Eaglescience ontwikkelaar
\item De module dient gescheiden componenten te bevatten: Frontend, Backend, API
\item Voor de API dient gebruik te worden gemaakt van swagger
\item De module dient goed gedocumenteerd te zijn middels 'in code comments'
\end{itemize}

%\subsection{Vereiste resultaten}\label{subsec:deliverables-vereiste-resultaten}
%Vanuit de CTO worden er naast de functionele eisen ook eisen gesteld aan de oplevering:
%\begin{itemize}
%\item Geïntegreerde en aantoonbaar werkende module
%\item De code van de module is gepubliceerd in Eaglescience GitLab
%\item Aanwezigheid van een handleiding over hoe de module gebruikt dient te worden
%\end{itemize}

\section{Gewenst neveneffect}\label{sec:gewenst-neveneffect}
Naast dat de nieuwe module inzicht moet geven in de kwetsbaarheden van bibliotheken van derden zal deze ook bewustzijn creëren in risico's van het gebruik hiervan. Voordat er zal worden overgegaan tot het gebruik van een bibliotheek dienen de volgende vragen te worden beantwoord:
\begin{itemize}
	\item Is deze bibliotheek echt nodig?
	\item Zo ja, welke invloed heeft dat op de veiligheid van de applicatie?
	\item Kan deze functionaliteit ook op een andere eenvoudige manier worden bewerkstelligd?
	\item Is er een andere bibliotheek die dezelfde functionaliteiten biedt?

\end{itemize}


\section{Opdracht fasen}\label{sec:opdracht-fasen}
Om de hierboven beschreven opdracht zo goed als mogelijk uit te voeren dient er eerst een onderzoek gedaan te worden naar begrippen binnen het domein SOUP, de ontwikkelomgeving van Eaglescience en daarnaast naar mogelijkheden om bibliotheken te screenen en te testen op kwetsbaarheden. Na de onderzoeksfase moet er een module ontwikkeld worden die deze mogelijkheid implementeert met inachtneming van de hierboven genoemde eisen.

\subsection{Fase 1: Onderzoek} \label{subsec:fase-1:-onderzoek}
Als eerste dient er onderzoek gedaan te worden naar de huidige situatie binnen Eaglescience waarbij er gekeken wordt naar de huidige dev-stack, de tooling, de werkwijze, als ook de huidige manier van uitrollen van applicaties. Daarna dient er een begrippen / literatuur onderzoek gedaan te worden binnen het domein SOUP om een goede kennis te vergaren over het domein welke als basis zal dienen voor de te implementeren module. Daarnaast dient er onderzoek gedaan te worden om te zien of er bibliotheken en resources zijn waar informatie over SOUP-bibliotheken te vinden is, en aan welke eisen deze bibliotheken moeten voldoen om kwetsbaar te worden bevonden. Deze fase wordt beschreven in het deel~\ref{prt:Onderzoek} van dit document.

\subsection{Fase 2: Oplevering SOUP analyse module}\label{subsec:fase-2:-oplevering-soup-analyse-module}
De met het onderzoek behaalde resultaten aangaande beschikbare resources om een SOUP-analyse uit te voeren vormen de leidraad voor de implementatie van de module. Deze module moet voldoen aan de eisen die gesteld zijn. Het ontwerp wordt beschreven in deel~\ref{prt:ontwerp}.

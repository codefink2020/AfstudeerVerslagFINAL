\chapter{Inleiding}\label{ch:impl-inleiding}
EagleScience is al een geruime tijd bezig met het verbeteren van de buildstraat en zaken te automatiseren om zo steeds efficiënter software uit te kunnen rollen. Daarnaast is het bouwen van veilige software één van de hoofdpunten waar bij EagleScience veel aandacht aan wordt besteed. Om dit te kunnen garanderen is iedere ontwikkelaar verplicht om te onderzoeken welke gevolgen het gebruik van bepaalde bibliotheken heeft op de ontwikkelde software. Dit onderzoek wordt op dit moment voor een groot deel handmatig gedaan door het uitpluizen van documentatie en registraties in een Vulnerability database op basis van de dependency declaraties in de ontwikkelde applicaties. Daarnaast is er een groei aan projecten binnen de Eaglescience gaande in zowel projecten als personeel. Zeker de groei aan projecten zorgt ervoor dat er steeds meet SOUP-analyses moet worden gedaan, wat weer meer tijd kost dat niet aan de ontwikkeling van de applicaties kan worden besteed. Een ander probleem is dat veel informatie die wordt verkregen middels deze analyses in eerste instantie alleen bij de teams zelf bekend is. Het laatste is dat op het moment dat projecten ``klaar`` zijn er niet meer actief wordt gekeken naar de gebruikte bibliotheken en er dus de mogelijkheid bestaat dat er kwetsbaarheden buiten ergens anders aan het licht komen die onopgemerkt blijven binnen Eaglescience. Zeker voor deze projecten is het van belang dat er periodiek een analyse wordt uitgevoerd waarbij de resultaten centraal worden opgeslagen.

Er is dus een behoefte aan een centraal systeem die de SOUP-analyses uitvoert op een periodieke en geautomatiseerde manier waarbij de resultaten centraal worden opgeslagen welke voor iedere geïnteresseerde met de juiste rechten beschikbaar is. In de komende delen zal er een ontwerp worden gepresenteerd voor dit systeem.

\chapter{Inleiding}\label{ch:impl-inleiding}
Eaglescience is al een geruime tijd bezig met het verbeteren van de buildstraat en zaken te automatiseren om steeds efficiënter software uit te kunnen rollen. Daarnaast is het bouwen van veilige software één van de hoofdpunten waar veel aandacht aan wordt besteed bij Eaglescience. Om dit te kunnen garanderen is iedere ontwikkelaar verplicht om te onderzoeken welke gevolgen het gebruik van bepaalde bibliotheken heeft op de ontwikkelde software. Op dit moment wordt dit onderzoek voor een groot deel handmatig gedaan. Hierbij worden documentatie en registraties in een Vulnerability database op basis van de dependency declaraties in de ontwikkelde applicaties uitgeplozen.
Eaglescience maakt momenteel een groei door in zowel het aantal projecten als in medewerkers. De groei in het aantal projecten zorgt ervoor dat er steeds meer SOUP-analyses moeten worden gedaan die veel tijd kosten welke niet aan de ontwikkeling van applicaties kan worden besteed. Daarnaast blijft de tijdens de analyse verkregen informatie in eerste instantie alleen bij de teams, en worden eventuele kwetsbaarheden in afgeronde projecten niet meer regelmatig geanalyseerd waardoor de mogelijkheid bestaat dat kwetsbaarheden ergens anders aan het licht komen die onopgemerkt blijven binnen Eaglescience. Zeker voor deze projecten is het van belang dat er periodiek een analyse wordt uitgevoerd waarbij de resultaten centraal worden opgeslagen.

Er is dus een behoefte aan een centraal systeem die de SOUP-analyses uitvoert op een periodieke en geautomatiseerde manier waarbij de resultaten centraal worden opgeslagen welke voor iedere geïnteresseerde met de juiste rechten beschikbaar is. In de komende delen zal worden ingegaan in zowel de theorie achter SOUP-Analyses en methoden die gebruikt kunnen worden om een analyse te doen om vervolgens een ontwerp aan te bieden voor een geautomatiseerd systeem dat deze analyses uit kan voeren.

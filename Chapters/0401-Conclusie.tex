% Chapter X

\chapter{Conclusie}\label{ch:conclusie} % Chapter title\label{ch:conclusie} % For referencing the chapter elsewhere, use \autoref{ch:name}
Eaglescience is een bedrijf dat complexe software oplossingen ontwikkeld op basis van de wensen van de klant. Dit doet het door met ongeveer 28 personen in scrum teams aan verschillende projecten simultaan te werken. Naast het voldoen aan de wensen van de klant is het leveren van veilige software een belangrijk aandachtspunt. Ook het verder automatiseren van bedrijfsprocessen welke getriggert worden door groei in zowel het aantal klanten als in personeel heeft er toe geleid om onderzoek te doen naar een mogelijkheid tot het analyseren van de door Eaglescience ontwikkelde software te onderzoeken op kwetsbaarheden. En omdat de meeste projecten gebruik maken van externe bibliotheken is er gekozen om een geautomitiseerde SOUP\footnote{Software of Unkown Provenance/Pedigree} analyse te ontwikkelen.




\section{Onderzoek}\label{sec:onderzoek}
V
\subsection{Theorie}\label{subsec:theorie}
Hoewel het bekend is dat het gebruik van bibliotheken niet altijd zonder gevaar is kan er door de huidige snelheid waarmee software ontwikkeld moet worden niet alles in eigen beheer worden geschreven. Hierdoor worden bedrijven gedwongen om gebruik te maken van externe bibliotheken. Eaglescience is hierop geen uitzondering. Uit onderzoek van onder andere Synopsys bleek echter dat er veel kwetsbaarheden in dit soort bibliotheken zit waarbij deze kwetsbaarheden vaak pas na iets meer dan 2 jaar worden verbeterd. Dit in tegen stelling tot het over het algemeen snel reageren, gemiddeld binnen 9 dagen, van de ontwikkelaars op een bekent geworden kwetsbaarheid. Er kan dus geconcludeerd worden dat er veel gewonnen kan worden op het gebied van veilige software in het detecteren van bekende kwetsbaarheden door gebruikers. Waarop vervolgens actie op kan worden ondernomen. Er zijn verschillende instanties die inzicht beogen te verschaffen in de kwetsbaarheden van bibliotheken. De NVD van NIST zorgt ervoor dat informatie beschikbaar is voor eenieder die hierin geïnteresseerd is. Daarnaast zorgt de OWASP  en het NCSC voor awareness en mogelijk tooling die gebruikt kan worden voor het onderzoeken van deze bibliotheken. De in dit hoofdstuk opgedane kennis zal worden gebruikt in het volgende hoofdstuk wat gaat over de zoektocht naar een methode om binnen Eaglescience kwetsbaarheden op te zoeken in externe bibliotheken.

\subsection{methode}\label{subsec:methode}
conclusie tests


eindconclusie onderzoek methode

In dit hoofdstuk zijn een aantal onderzoeken uitgevoerd. Deze hebben inzichten opgeleverd over hoe Eaglescience software ontwikkeld, waarbij aandacht is besteed aan de dev-stack, werkwijze en tooling. Deze bevindingen vormden de basis voor de zoektocht naar compatibele tooling voor SOUP analyses binnen Scala en TypeScript projecten, waarbij SBT en NPM worden gebruikt als buildtools. Doordat Eaglescience ontwikkeld in een 'niche-taal' (Scala) is de beschikbare tooling gelimiteerd. Desondanks is er voor beide door Eaglescience gebruikte hoofdtalen geschikte tooling gevonden om SOUP analyses mee uit te voeren (OWASP-dependency-check voor NPM en sbt-dependency-check voor SBT). Uit documentatie van deze tooling bleek dat er testen mee konden worden uitgevoerd om te onderzoeken hoe ze in een methode zouden kunnen worden geïmplementeerd om aan de project eisen te kunnen voldoen. De geselecteerde tooling bleken compatibel en goed te werken tijdens een reeks testen, en te voldoen aan de in de opdracht gestelde eisen. Tijdens deze testen werd inspiratie opgedaan voor een methode om gegevens uit de Jenskins buildstraat te verkrijgen en te analyseren. Door samenvoeging van de twee testmethodes is er een methode gevonden voor de analyse van externe bibliotheken binnen Eaglescience projecten. Deze methode maakt het in theorie mogelijk om zowel na deploy geautomatiseerd te draaien als op periodieke basis. Echter is de methode die is beschreven geschikt voor SBT en NPM projecten. Het doel is dat de output uit de buildstraat opgepikt moet kunnen worden door de API die vervolgens verantwoordelijk is voor het verwerken van de rauwe data in een gewenst formaat. Ieder platform dat wordt toegevoegd aan de SOUP analyse heeft dus een eigen SCA tool nodig met daarbij een passende plugin die de data van deze tool omzet naar het datamodel dat wordt gehanteerd binnen de API. De bevindingen van dit hoofdstuk zullen als basis dienen voor het ontwerp van de daadwerkelijke module voor de uitvoering van SOUP analyses.


%----------------------------------------------------------------------------------------

\section{Ontwerp}\label{sec:implementatie}

Content

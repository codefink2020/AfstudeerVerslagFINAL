% Chapter X

\chapter{Conclusie}\label{ch:conclusie} % Chapter title\label{ch:conclusie} % For referencing the chapter elsewhere, use \autoref{ch:name}

Het leveren van veilige software is naast het voldoen aan de wensen van de klant een belangrijk vereiste voor Eaglescience. Tijdens softwareontwikkeling maakt Eaglescience gebruik van externe bibliotheken welke kwetsbaarheden kunnen bevatten. Door de groei in het aantal klanten is het automatiseren van bedrijfsprocessen, waaronder analyse van ontwikkelde software op kwetsbaarheden, noodzakelijk geworden. Het hier beschreven afstudeerproject onderzocht mogelijkheden tot het periodiek geautomatiseerd uitvoeren van SOUP-analyses geschikt voor de specifieke werksituatie binnen Eaglescience.
Het onderzoek bestond uit een aantal fasen. Allereest is de theorie over SOUP in kaart gebracht waarbij aandacht is besteed aan welke gevaren er potentieel kunnen worden verwacht bij het gebruik van externe bibliotheken, alsook de instanties die zich bezig houden met het bijhouden van kwetsbaarheden in voor iedereen toegankelijke databases. Uit dit onderzoek kon worden geconcludeerd dat het gebruik van externe bibliotheken onvermijdelijk is geworden om aan de huidige ontwikkelsnelheid te kunnen voldoen en dat volledige ontwikkeling in eigen beheer te kostbaar is. Externe bibliotheken bevatten echter kwetsbaarheden die vaak pas na lange tijd worden ontdekt wanneer hier niet op regelmatige basis op wordt gescreend. Organisaties zoals OWASP spelen een hoofdrol in het inzichtelijk maken van kwetsbaarheden.
Om een SOUP-analyse applicatie te kunnen ontwerpen voor de specifieke werksituatie van Eaglescience is in het hierop volgende deelonderzoek in kaart gebracht welke dev-stack, werkwijze en tooling Eaglescience gebruikt bij de ontwikkeling van software. Hieruit kon worden geconcludeerd dat de SOUP-analyse compatibel moet zijn met Scala en TypeScript projecten, waarbij SBT en NPM als build-tooling wordt gebruikt.

Om analyses uit te kunnen voeren moet er Software Composition Tooling (SCA) worden gevonden die het mogelijk maakt om deze analyses uit te voeren. Vooral het vinden van een geschikte SCA Tool voor gebruik binnen Scala projecten bleek gelimiteerd. Uit het onderzoek kon worden geconcludeerd dat OWASP-dependency-check voor NPM en sbt-dependency-check voor SBT het meest geschikt waren. Na uitvoering van een reeks testen met deze tooling kon worden geconcludeerd dat deze toe te voegen is aan de tooling binnen Eaglescience en dan met name de Jenkins buildstraat. Daarnaast komt het resultaat uit de beide tools overeen met elkaar wat betreft het schema van de JSON.

Op basis van de uit dit deelonderzoek getrokken conclusies werd een methode voor automatische SOUP-analyse ontwikkeld, waarbij gegevens uit de Jenskins build straat worden verkregen om te analyseren. De ontwikkelde methode maakt het mogelijk om middels een dependency check zowel tijdens een deploy automatisch als na een deploy op  periodieke basis een SOUP-analyse uit te voeren doordat projectgegevens veilig worden gesteld. De SOUP-analyse resulteert in een uitgebreid rapport over de eventueel gevonden kwetsbaarheden.

Op basis van dit afstudeerverslag kan worden geconcludeerd dat de hier beschreven methode het mogelijk zou maken om inzichten te verkrijgen in kwetsbaarheden geïntroduceerd door het gebruik van externe bibliotheken. Door implementatie van zo’n applicatie zou Eaglescience de mogelijk hebben om deze inzichten te gebruiken zodat preventie stappen beschreven in OWASP top 10 - A06:2021 $”$Vulnerable and outdated components$"$ kunnen worden nageleefd. Op het moment van schrijven worden de SOUP-analyses nog handmatig en hierdoor op een lage frequentie uitgevoerd. Implementatie van de hier ontworpen automatische methode zou om een minder grote tijdsinvestering vragen dan momenteel het geval is. Door het op periodieke basis inzetten van SOUP-analyses zouden kwetsbaarheden in een vroegtijdig stadium kunnen worden ontdekt, waardoor hierop snel actie kan worden ondernomen. Dit zou dit veiligheid van software welke Eaglescience ontwikkeld ten goede komen.

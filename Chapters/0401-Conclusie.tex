% Chapter X

\chapter{Conclusie}\label{ch:conclusie} % Chapter title\label{ch:conclusie} % For referencing the chapter elsewhere, use \autoref{ch:name}
Eaglescience is een bedrijf dat complexe software oplossingen ontwikkeld op basis van de wensen van de klant. Dit doet het door met ongeveer 28 personen in scrum teams aan verschillende projecten simultaan te werken. Naast het voldoen aan de wensen van de klant is het leveren van veilige software een belangrijk aandachtspunt. Ook het verder automatiseren van bedrijfsprocessen welke getriggert worden door groei in zowel het aantal klanten als in personeel heeft er toe geleid om onderzoek te doen naar een mogelijkheid tot het analyseren van de door Eaglescience ontwikkelde software te onderzoeken op kwetsbaarheden. En omdat de meeste projecten gebruik maken van externe bibliotheken is er gekozen om een geautomitiseerde SOUP\footnote{Software of Unkown Provenance/Pedigree} analyse te ontwikkelen.




\section{Onderzoek}\label{sec:onderzoek}
Voordat er aan een implementatie begonnen kon worden moest eeste de theorie van SOUP worden onderzocht waarbij aandacht is besteed aan de welke gevaren er potentieel kunnen worden verwacht. Alsook de instanties die zich bezig houden met het bijhouden van kwetsbaarheden in voor iedereen toegankelijke databases.
\subsection{Theorie}\label{subsec:theorie}


\subsection{methode}\label{subsec:methode}

Content

%----------------------------------------------------------------------------------------

\section{Ontwerp}\label{sec:implementatie}

Content

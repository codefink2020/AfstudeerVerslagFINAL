\chapter{Functioneel ontwerp: Portal}\label{ch:impl-Portal}
De SOUP-API is de centrale module voor het systeem die verantwoordelijk is voor het periodiek scannen van de projecten en het parsen van de rapporten die uit de analyses komt. Deze module is ook opgedeeld in twee verschillende submodulen die elk in de komende secties verder worden uitgewerkt.


\section{Frontend}
In de frontend moeten er onderdelen worden toegevoegd aan de huidige project module waarin een samenvatting is te zien over de gevonden kwetsbaarheden. Om de details van de analysen te kunnen zien moet er een aparte module komen waarin deze details te zien zijn. In de SOUP module moet de mogelijkheid komen om per project per module inzicht te verkrijgen in de gevonden kwetsbaarheden.


\section{Backend}
Ook hier moet in de project module een aanpassing worden gemaakt in de vorm van een aantal endpoints die de door de portal opgevraagde data kan verzenden en/of wijzigen. Voor de SOUP module zelf is er gekozen om direct met de datacontroller van de SOUP-API te communiceren.

\section{Portal Database}
Hier moeten een aantal aanpassingen op worden gedaan die in het architectuur hoofdstuk zijn besproken in sectie ~\ref{subsec:portal-datamodel}.

'


\section{Portal}\label{sec:portal}
In de portal wordt er een module toegevoegd die als interface dient voor het SOUP-analyse systeem. Informatie over kwetsbaarheden in geanalyseerde projecten worden hier weergegeven. Ook is het hier mogelijk om per project in te kunnen zien welke instellingen er gelden voor analyses en deze kunnen hier indien nodig worden aangepast.
In de module moet er dus een plek komen waar informatie betreft kwetsbaarheden kan worden geraadpleegd. Daarnaast moet er een plek komen waarbij instellingen kunnen worden aangepast zodat de analyses worden uitgevoerd op de ingestelde periode en/of aan en uit gezet kan worden.
